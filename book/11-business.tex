\chapter{Bussiness}


% ---------- Bussiness Sample Metric ----------
\clearpage
\thispagestyle{businessstyle}
\section{Bussiness Sample Metric}
\subsection{Bussiness Sample Metric}

% ---------- CR ----------
\clearpage
\thispagestyle{businessstyle}

\section{CR}

\subsection{Conversion Rate}
The conversion rate is the percentage of people who visit a website or app and then take a specific action, like buying something or subscribing to a newsletter, after seeing a prompt or marketing message. It helps measure how well those prompts turn visitors into customers or leads.

% equation
\begin{center}
    \tikz{
        \node[inner sep=2pt, font=\large] (a) {
            {
                $\displaystyle
                 Conversion \, Rate =\frac{Number \, of \,Conversions}{Number \, of \, Visits}\times100
                $
            }
        };
    }
\end{center}

\vspace{-10pt}

where desired outcomes are the actions taken by users (like purchases or sign-ups), and total attempts represent the number of opportunities for those actions (such as visits or clicks)

\textbf{When to use Conversion Rate?}

Use the conversion rate (CR) to evaluate marketing effectiveness, website performance, and user experience optimization, especially during campaigns, A/B testing, and product launches.

\coloredboxes{  % Ensure that this command is defined in your preamble
\item It is a straightforward and widely accepted standard for comparing vision algorithms.
\item Simple and fast to compute.
}
{
\item It doesn't explain why users convert.
\item It may not reflect customer retention.
}

\textbf{Other related metrics}

Other related metrics to Conversion Rate include Bounce Rate and Return on Investment (ROI).

\vspace{-10pt}

\orangebox{%
Did you know that...}
{
    Adobe redesigned its website to enhance user experience and targeted messaging, resulting in a \%20 increase in conversion rates. This improvement led to an additional \$1 billion in annual revenue.
}
