\chapter{Classification}
\section{FPR}
\subsection{False Positive Rate}
\thispagestyle{classificatiostyle}

The False Positive Rate (FPR), also known as the false alarm ratio or fall-out, measures how often negative instances are incorrectly classified as positive in binary classification.

\begin{center}
\tikz{
\node[inner sep=2pt, font=\Large] (a) {
{
$\displaystyle
FPR = \frac{{\color{cyan}FP}}{{\color{cyan}FP} + {{\color{nmlpurple}TN}}}
$
}
};
\draw[-latex,cyan, semithick] ($(a.north)+(1.3,0.05)$) to[bend left=15] node[pos=1, right] {False positives} +(1,.5); 
% \draw[-latex,teal!70!green, semithick] ($(a.south)+(2.1,0.1)$) to[bend right=15] node[pos=1, right] {Mean of targets} +(1,-.5); 
\draw[-latex,nmlpurple, semithick] ($(a.south)+(1.5,-0.05)$) to[bend left=15] node[pos=1, left] {True negatives} +(-1,-.5); 
}
\end{center}

FPR ranges from 0 (no false alarms) to 1 (all predicted positives are incorrect). FPR can also be interpreted as the probability that a negative instance will be incorrectly identified as positive.

\textbf{When to use FPR?}

Use FPR when you need to evaluate how well a classifier avoids false positives, especially when false positives have significant costs, like in medical diagnostics or security systems. It's also useful for understanding the trade-off between true positive rate (sensitivity) and false positive rate.

\coloredboxes{
\item It provides a clear and intuitive measure of a classifier's false positive performance.
\item It helps identify scenarios where the classifier is overly sensitive and prone to false alarms.
}
{
\item FPR does not consider true positive instances.
\item FPR can be sensitive to class imbalance, as it may be easier to achieve a low FPR when the negative class is dominant.
\item FPR doesn't exist in isolation; it's often important to show its relationship with another key metric. (e.g., TPR, Precision, Recall).
}


\clearpage
\thispagestyle{customstyle2}


\begin{figure*}[ht!]
    \centering
    \includegraphics[width=0.7\textwidth]{figures/FPR_3d_surface.png}
    % \caption{Caption}
    \label{fig1}
\end{figure*}

\begin{wrapfigure}{r}{0.55\textwidth}
    \centering
    \vspace{-20pt} % Adjust vertical alignment if needed
    \includegraphics[width=0.5\textwidth]{figures/FPR_2d_line_plot.png} % Your figure goes here
\end{wrapfigure}

% Left text with the image on the right
\textbf{Figure 3.1 False Positive Rate.} 
\textbf{Top:}
3D surface illustrating FPR's non-linear relationship with FP and TN. FPR is lowest (blue) when FP is low. It increases (red) as FP increases.
\textbf{Right:}
Shows how FPR decreases hyperbolically as total negative cases increase for fixed FP values. Lower FP maintains better FPR.


\orangebox{%
Did you know that...}
{
In the context of statistical hypothesis testing, the FPR is also known as the "type I error rate" or the probability of rejecting a true null hypothesis.
}


\textbf{FPR alternatives}

Other metrics used alongside or instead of FPR include True Positive Rate (TPR), Precision, F1-Score, Receiver Operating Characteristic (ROC AUC), and Specificity.