\chapter{Bias \& Fairness}


% ---------- Demographic Parity ----------
\clearpage
\thispagestyle{biasfairnesstyle}
\section{Demographic Parity}
\subsection{Demographic Parity}

% reference:
% https://developers.google.com/machine-learning/crash-course/fairness/demographic-parity
% https://afraenkel.github.io/fairness-book/content/05-parity-measures.html#demographic-parity
% https://fairmlbook.org/classification.html

Demographic Parity (also known as Statistical Parity) is one of the most widely cited fairness metrics in machine learning.
It requires that the decision of a model be independent of a protected attribute (such as gender, race, or age). In other words,
the probability of receiving a positive prediction should be the same across groups, regardless of whether the prediction is correct.

% Formula
\begin{center}
    % P(Y^=1∣A=a)=P(Y^=1∣A=b)∀a,b
    FORMULA GOES HERE
\end{center}

A model satisfies demographic parity if members of different demographic groups are selected at equal rates. For example, in a loan approval
system, demographic parity would mean that the proportion of approved applications is the same across groups, independent of creditworthiness.

\textbf{When to use Demographic Parity?}

Use demographic parity when the fairness objective is equal treatment across groups in terms of outcomes, even if this comes at the cost
of predictive accuracy. It is often applied in contexts such as hiring, lending, or college admissions, where ensuring equal access is a
primary concern.

\coloredboxes{
\item Simple to compute and easy to explain to non-technical stakeholders.
\item Guarantees equal selection rates across groups, supporting inclusion.
}
{
\item It only looks at whether groups get the same share of positive outcomes, but not at whether those predictions are actually correct.
\item Can reduce overall model accuracy if the underlying distributions differ.
}

\clearpage

\thispagestyle{customstyle}

\orangebox{Did you know that...}
{Demographic parity goes by several other names in the literature, often referred to as statistical parity, group fairness, or even
independence criterion. Different research communities picked different terms, but they all describe the same idea.}