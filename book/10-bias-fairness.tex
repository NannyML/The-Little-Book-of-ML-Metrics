\chapter{Bias \& Fairness}


% ---------- Demographic Parity ----------
\clearpage
\thispagestyle{biasfairnesstyle}
\section{Demographic Parity}
\subsection{Demographic Parity}

% reference:
% https://developers.google.com/machine-learning/crash-course/fairness/demographic-parity
% https://afraenkel.github.io/fairness-book/content/05-parity-measures.html#demographic-parity
% https://fairmlbook.org/classification.html

Demographic Parity (also known as Statistical Parity) is one of the most widely cited fairness metrics in machine learning.
It requires that the decision of a model be independent of a protected attribute (such as gender, race, or age). In other words,
the probability of receiving a positive prediction should be the same across groups, regardless of whether the prediction is correct.

% Formula
\begin{center}
    % P(Y^=1∣A=a)=P(Y^=1∣A=b)∀a,b
    FORMULA GOES HERE
\end{center}

A model satisfies demographic parity if members of different demographic groups are selected at equal rates. For example, in a loan approval
system, demographic parity would mean that the proportion of approved applications is the same across groups, independent of creditworthiness.

\textbf{When to use Demographic Parity?}

Use demographic parity when the fairness objective is equal treatment across groups in terms of outcomes, even if this comes at the cost
of predictive accuracy. It is often applied in contexts such as hiring, lending, or college admissions, where ensuring equal access is a
primary concern.

\coloredboxes{
\item Simple to compute and easy to explain to non-technical stakeholders.
\item Guarantees equal selection rates across groups, supporting inclusion.
}
{
\item It only looks at whether groups get the same share of positive outcomes, but not at whether those predictions are actually correct.
\item Can reduce overall model accuracy if the underlying distributions differ.
}

\clearpage

\thispagestyle{customstyle}

\orangebox{Did you know that...}
{Demographic parity goes by several other names in the literature, often referred to as statistical parity, group fairness, or even
independence criterion. Different research communities picked different terms, but they all describe the same idea.}


% ---------- Equality of Opportunity ----------
\clearpage
\thispagestyle{biasfairnesstyle}
\section{Equality of Opportunity}
\subsection{Equality of Opportunity}

Equality of Opportunity is a fairness metric that focuses on ensuring that individuals who truly belong to the positive class
(e.g., qualified applicants) have an equal chance of being correctly identified across different demographic groups.
Formally, the metric requires that the True Positive Rate (TPR) be equal across groups.

% Formula
\begin{center}
    % P(Y^=1∣Y=1,A=a)=P(Y^=1∣Y=1,A=b)
    FORMULA GOES HERE
\end{center}

In other words, if two people are equally qualified, their chance of being recognized as such by the model should not depend on their
demographic group.

\textbf{When to use Equality of Opportunity?}

This metric is particularly useful in high-stakes scenarios where false negatives carry large consequences, such as university admissions,
medical diagnoses, or loan approvals. It ensures that qualified individuals are treated equally, regardless of group membership.

\coloredboxes{
\item Ensures qualified individuals have equal chances across groups, even if overall acceptance rates differ.
\item Models can vary in their ratio of positive to negative predictions across groups, as long as the true positives are treated fairly.
\item Can be more aligned with fairness in practical scenarios, where ensuring equal opportunity for qualified candidates
is more meaningful than enforcing equal acceptance rates.
}
{
\item Only applies when there is a clearly preferred label (e.g., “qualified”), limiting its use to such contexts.
\item Does not guarantee fairness for the negative class, unlike metrics such as Equalized Odds.
\item Requires demographic group labels to compare outcomes. If group membership is unavailable, the metric cannot be applied.
}

\clearpage

\thispagestyle{customstyle}

\orangebox{Did you know that...}
{Equality of Opportunity was popularized in the 2016 paper “Equality of Opportunity in Supervised Learning” by Hardt, Price, and Srebro. 
In that paper, they introduced both Equality of Opportunity and its stricter sibling, Equalized Odds. The terms have since become standard
in the fairness in ML literature.}